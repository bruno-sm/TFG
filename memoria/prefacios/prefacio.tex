\chapter*{}
%\thispagestyle{empty}
%\cleardoublepage

%\thispagestyle{empty}

%\input{portada/portada_2}


\newpage
\thispagestyle{empty}

\begin{center}
{\large\bfseries \myOfficialTitle}\\
\end{center}
\begin{center}
\myName\\
\end{center}

%\vspace{0.7cm}
\noindent{\textbf{Palabras clave}: \keywords}\\

\vspace{0.7cm}
\noindent{\textbf{Resumen}}\\

La teoría de lenguajes de programación es una rama de las ciencias de la computación que nació por motivos meramente prácticos. Cuando se comenzaron a crear los primeros lenguajes para sustituir a la programación en ensamblador se hizo evidente la necesidad de modelos teóricos con los que describir y estudiar los complejos sistemas necesarios para su desarrollo. Valiéndose de la teoría de autómatas, la teoría de tipos y los modelos de computación, este es un campo amplio que ejemplifica perfectamente la colaboración que puede darse entre las matemáticas y la informática. Este TFG puede dividirse en una parte teórica, desarrollada en los capítulos \ref{sect:tt} y \ref{sect:sistemas_de_calculo}, y una parte práctica, presente en los capítulos \ref{sect:ap_ia}, \ref{sect:form_tail} y \ref{sect:impl}.\\

En el apartado teórico se estudian la teoría de tipos y los sistemas de cálculo, mientras que en el práctico, con la intención de diseñar un lenguaje especializado en inteligencia artificial, se estudian cual deberían ser sus aspectos fundamentales, se formaliza su gramática, semántica y sistema de tipos, se demuestra su buen comportamiento y, por último, se implementa parcialmente su compilador.\\
\cleardoublepage


\thispagestyle{empty}


\begin{center}
{\large\bfseries \myOfficialTitleEng}\\
\end{center}
\begin{center}
\myName\\
\end{center}

%\vspace{0.7cm}
\noindent{\textbf{Keywords}: \keywordsEng}\\

\vspace{0.7cm}
\noindent{\textbf{Abstract}}\\

As a branch of computer science, programming language theory was developed mostly for practical reasons. When the first programming languages began to emerge in order to replace assembler, was noticeable that a theoretic model was needed to describe and study the complex systems involved in its development. With the adoption of automata theory, type theory and computability theory this is a broad area and a paradigm in the combination of maths and computer science. This work can be divided in a theoretical section, represented by chapters \ref{sect:tt} and \ref{sect:sistemas_de_calculo} and a more practical one, developed in chapters \ref{sect:ap_ia}, \ref{sect:form_tail} y \ref{sect:impl}.\\

In the theoretical section we study some basic type theory and calculi systems, meanwhile in the practical section, with the purpose of design a programming language specialized in artificial intelligence, we discuss what should be its foundational components, formalize its grammar, semantics and type system, prove its good behavior and at last a compilator is partially implemented.\\

\chapter*{}
\thispagestyle{empty}

\noindent\rule[-1ex]{\textwidth}{2pt}\\[4.5ex]

Yo, \textbf{\myName}, alumno de la titulación Doble Grado en Ingeniería Informática y Matemáticas de la \textbf{Escuela Técnica Superior
de Ingenierías Informática y de Telecomunicación de la Universidad de Granada}, con DNI 71304914S, autorizo la
ubicación de la siguiente copia de mi Trabajo Fin de Grado en la biblioteca del centro para que pueda ser
consultada por las personas que lo deseen.

\vspace{6cm}

\noindent Fdo: \myName

\vspace{2cm}

\begin{flushright}
Granada a 05 de mes 09 de 2019 .
\end{flushright}


\chapter*{}
\thispagestyle{empty}

\noindent\rule[-1ex]{\textwidth}{2pt}\\[4.5ex]

D. \textbf{\myProf}, Profesor del Departamento de Lenguajes y Sistemas Informáticos de la Universidad de Granada.

\vspace{0.5cm}


\textbf{Informa:}

\vspace{0.5cm}

Que el presente trabajo, titulado \textit{\textbf{\myOfficialTitle}},
ha sido realizado bajo su supervisión por \textbf{\myName}, y autoriza la defensa de dicho trabajo ante el tribunal
que corresponda.

\vspace{0.5cm}

Y para que conste, expide y firma el presente informe en Granada a 05 de mes 09 de 2019 .

\vspace{1cm}

\textbf{El director:}

\vspace{5cm}

\noindent \textbf{\myProf}

\chapter*{Agradecimientos}
\thispagestyle{empty}

       \vspace{1cm}


Quizá peque de clásico, pero no puedo dejar de agradecer el enorme apoyo recibido por parte de mi familia durante la realización de este trabajo, ayudándome en mis otras responsabilidades y haciéndome la tarea mucho más llevadera. Quiero hacer una mención especial a mi madre, sin ella ni siquiera hubiera empezado a estudiar este grado. También a mi hermana, por el diseño de un genial logo para el proyecto, un toque de distinción del que no estoy seguro que esté a la altura.\\

Por supuesto, mi segundo pensamiento está dedicado al prof. Ramón López-Cózar Delgado, por haber aceptado ser mi Tutor para la realización de este trabajo y haber demostrado una gran disponibilidad para asesorarme en todo momento.\\



