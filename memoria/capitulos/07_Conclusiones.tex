\chapter{Conclusiones y trabajo futuro}
\label{sect:con}

En este trabajo hemos abordado elementos fundamentales de la teoría de lenguajes de programación. Hemos visto como se relaciona la teoría de tipos matemática con su aplicación en lenguajes, mediante el estudio de los sistemas de tipado y su capacidad para codificar el funcionamiento de sistemas tipos que comprueban el buen uso de los tipos teóricos.\\

También se han presentado técnicas fundamentales como la demostración de la seguridad de un lenguaje mediante los teoremas de progreso y preservación y la comprobación de la turing-completitud haciendo uso de las funciones $\mu$-recursivas.\\

Se ha demostrado que el análisis de un lenguaje es una fase fundamental en su diseño y que los trabajos realizados en este campo dan como resultado mejoras tangibles en la calidad de los lenguajes que utilizamos diariamente. Sin embargo, aún queda mucho espacio para profundizar en los aspectos teóricos. Una vía de investigación futura interesante sería el estudio de la teoría de tipos dependientes, tanto sus posibles aplicaciones como sus implicaciones teóricas y sus conexiones con la lógica mediante el isomorfismo de Curry-Howard.\\

Respecto a la parte práctica, hemos diseñado un nuevo lenguaje, identificando los factores que lo pueden hace útil al usarlo en inteligencia artificial. Se ha formalizando su sintaxis, semántica y sistema de tipos y se han aplicado los resultados teóricos aprendidos para demostrar su buen comportamiento. A su vez se ha realizado una implementación parcial de su compilador, aprendiendo por el camino que antes de invertir tiempo en una tecnología es necesario cerciorarse de que se adecúa a nuestras necesidades. Una línea de trabajo evidente sería la conclusión del compilador, eligiendo un método de generación de código más adecuado. Otros elementos interesantes para desarrollar sería el diseño de un sistema de módulos para \textit{tail} y la adición de facilidades para la programación concurrente.\\